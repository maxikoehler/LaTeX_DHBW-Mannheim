%%**************************************************************
%% Vorlage fuer Bachelorarbeiten (o.ä.) der DHBW
%%
%% Autor: Tobias Dreher, Yves Fischer, Michael Gruben, Markus Barthel
%% Datum: 06.07.2011 - 22.08.2014
%% 
%% Autor: Ferdinand König und Maximilian Köhler
%% Datum: 2020 - 2022
%%**************************************************************

\input{ads/header}

\makeglossaries

\begin{filecontents}[overwrite]{images/appendix/\jobname-vergleich-ess.tex}
    \begin{longtable}{XXXX}
        \small \singlespacing
        \\ \hline \rowcolor{lightgray} 
        A & B & C & D \\ \hline \endhead
        % \hline \rowcolor{lightgray} Energiespeicher     & Vorteile & Nachteile & Anwendungsszenarien \hline \endhead
        % \hline \endfoot
        % \midrule
        % \hline \endlastfoot
        Eine paar sehr lange Spalten um einen Seitenumbruch zu erzwingen & & & \\ \hline
        Eine paar sehr lange Spalten um einen Seitenumbruch zu erzwingen & & & \\ \hline
        & & & \\ \hline
        Eine paar sehr lange Spalten um einen Seitenumbruch zu erzwingen & & & \\ \hline
        \rowcolor{light-green} \textbf{ABC} & & & \\ \hline
        Eine paar sehr lange Spalten um einen Seitenumbruch zu erzwingen & & & \\ \hline
        Eine paar sehr lange Spalten um einen Seitenumbruch zu erzwingen & & & \\ \hline
        Eine paar sehr lange Spalten um einen Seitenumbruch zu erzwingen & & & \\ \hline
        & & & \\ \hline
        & & & \\ \hline
    \end{longtable}
\end{filecontents} 

\begin{filecontents}[overwrite]{images/Zeitverlauf2.csv}
    Jahr;Anmeldungen;Erteilungen
    2006;2;0
    2007;3;0
    2008;1;0
    2009;38;0
    2010;11;0
    2011;30;2
    2012;107;5
    2013;238;6
    2014;238;10
    2015;339;19
    2016;513;45
    2017;666;83
    2018;728;139
    2019;818;170
    2020;755;310
    2021;491;708
    2022;136;504
\end{filecontents}

\begin{filecontents}[overwrite]{images/Tabelle2.csv}
    A;B;C
    0.100070106;1.568510679;0.156961029
    0.303991153;1.566032176;0.476059928
    0.506915855;1.56336302;0.792493502
    1.013237861;1.559740594;1.580388223
    2.027175145;1.554211627;3.150659179
    5.06845253;1.540103229;7.805940107
    7.598314008;1.529426603;11.62106358
    10.12963371;1.519131286;15.38824349
    15.20682242;1.501591115;22.83442944
    20.27091346;1.485194869;30.10625666
    25.34373408;1.469370585;37.23933736
    30.39909554;1.454308917;44.20967571
    35.47628425;1.439437903;51.06590819
    40.54910486;1.424757543;57.772643
    45.62046725;1.410649145;64.35447311
    50.66498108;1.397875325;70.82332689
    59.08267434;1.37537815;81.26101931
    67.55330591;1.353643591;91.44309957
    76.02394408;1.331337069;101.2134949
    84.45928124;1.309793164;110.6241892
    92.8769745;1.285961411;119.4362052
    101.2593733;1.265370776;128.1306518
    109.8182442;1.242301639;136.4273848
    118.2182936;1.220757734;144.3158962
    126.6712813;1.197497943;151.6885987
    135.1242689;1.175572729;158.8484056
    143.5419622;1.152503592;165.4326271
    151.9949498;1.129434455;171.6683333
    160.5185197;1.107127934;177.7145371
    168.9185691;1.084821413;183.2464807
    177.3539062;1.063086854;188.5426062
    185.8068939;1.03830183;192.923638
    194.1892928;1.012754191;196.66602
    202.6952253;0.986825243;200.024765
    211.1835074;0.955748637;201.8383493
    219.6188446;0.929247727;204.0803121
    228.0718322;0.902937471;205.9346034
    236.4365872;0.875673945;207.0413591
    244.9425132;0.848982382;207.9518782
    278.6485806;0.71209279;198.4236453
    303.9898997;0.573487313;174.3343507
    329.3665066;0.38207067;125.8412818
    354.6548808;0.137080245;48.61617805
\end{filecontents}

\begin{document}
	% Deckblatt
	\begin{spacing}{1}
		\input{ads/deckblattSA}
	\end{spacing}
	\newpage

	% stellt Abstand vor Kapitelüberschriften ein
	\RedeclareSectionCommand[beforeskip=\kapitelabstand         ]{chapter}
	\newgeometry{left=4cm,right=2.5cm,top=2.5cm,bottom=2.5cm}

	\pagenumbering{Roman}
	\clearpairofpagestyles
	\ohead[]{\headmark}				% Kopfzeile außen immer mit Headmark versehen
	\automark[section]{chapter}		% Headmark bestehend aus Kolumnentitel
	\ofoot[\pagemark]{\pagemark}	% Fußzeile mit Seitenzahl außen
	\renewcommand*\chapterpagestyle{plain.scrheadings}
	% \renewcommand*\partpagestyle{plain.scrheadings}		%Bei Verwendung von Parts als Überschriftenebene: Setzen des Pagestyles global
    
    \setcounter{page}{2}
	% Sperrvermerk
	%!TEX root = ../main.tex

% \newgeometry{left=2.5cm,right=2.5cm,top=2.5cm,bottom=2.5cm}

% Sperrvermerk direkt hinter Titelseite

\addchap*{\langsperrvermerk}
\thispagestyle{empty}

\vfill

\begin{figure}[H]
    \centering
    \href{https://www.curemannheim.de}{\includegraphics[height=5cm]{images/essential/firmenlogo.png}}
\end{figure}
\vspace*{2cm}

\iflang{de}{
Der Inhalt dieser Bachelorarbeit darf weder als Ganzes noch in Auszügen Personen außerhalb des Prüfungsprozesses und des Evaluationsverfahrens zugänglich gemacht werden, sofern keine anders lautende Genehmigung von CURE vorliegt.\\

\vspace*{1cm}
Insbesondere ist eine Weitergabe an Wettbewerber von CURE oder eine Veröffentlichung, auch auszugsweise, nicht gestattet.\\

\vspace*{1cm}
Ausnahmen bedürfen der schriftlichen Genehmigung durch den entsprechenden Hauptabteilungsleiter von CURE.
}

\iflang{en}{%
  The {\arbeit} on hand 
  \begin{center}{\itshape{} \titel{}\/}\end{center} 
   contains internal resp.\ confidential data of {\firma}. It is intended solely for inspection by the assigned examiner, the head of the {\studiengang} department and, if necessary, the Audit Committee \langanderdh{} {\dhbw}. It is strictly forbidden
    \begin{itemize}
    \item to distribute the content of this paper (including data, figures, tables, charts etc.) as a whole or in extracts,
    \item to make copies or transcripts of this paper or of parts of it,
    \item to display this paper or make it available in digital, electronic or virtual form.
    \end{itemize}
  Exceptional cases may be considered through permission granted in written form by the author and {\firma}.
}

\vspace{4cm}

% \restoregeometry
    \newpage
	
	% Erklärung
 	%!TEX root = ../main.tex

% \newgeometry{left=2.5cm,right=2.5cm,top=2.5cm,bottom=2.5cm}

\addchap*{\langerklaerung}
\thispagestyle{empty}

\vspace*{1.5cm}

\iflang{de}{
    \begin{center}
        \begin{tabular}{| p{0.95\textwidth} |}
            \hline
            Ich versichere hiermit, dass ich meine \arbeit~mit dem Thema \glqq{\itshape \titel }\grqq~selbstständig verfasst und keine anderen als die angegebenen Quellen und Hilfsmittel benutzt habe.\\
            Ich versichere zudem, dass die eingereichte elektronische Fassung mit der gedruckten Fassung übereinstimmt.*\\
            \footnotesize{\emph{*Falls beide Fassungen gefordert sind}}\\
            \vspace{.5cm}
            Mannheim, den \datumAbgabe\\
            \vspace*{.5cm}
            \singlespacing
            \rule{7cm}{.5pt}\\
            \autor\\[12pt]
            \hline
        \end{tabular}
    \end{center}

    \vfill

    \begin{flushright}
        \begin{minipage}[]{0.8\textwidth}
            \flushright
            \textbf{Hinweis:}\\[6pt]
            In dieser \arbeit~wird aus Gründen der besseren Lesbarkeit das generische Maskulinum verwendet. Weibliche und anderweitige Geschlechteridentitäten werden dabei ausdrücklich mitgemeint, soweit dies für die Aussage erforderlich ist.
        \end{minipage}
    \end{flushright}
}


\iflang{en}{
    % Müsste man mal ausfüllen für eine englische Arbeit.
}

% \restoregeometry
 	\newpage

	% Abstract
	%!TEX root = ../main.tex

\pagestyle{empty}


\renewcommand{\abstractname}{\langabstract} % Text für Überschrift

\begin{otherlanguage}{english} % auskommentieren, wenn Abstract auf Deutsch sein soll
\begin{abstract}

\begin{description}
\item[Objektivität] soll sich jeder persönlichen Wertung enthalten
\item[Kürze] soll so kurz wie möglich sein
\item[Genauigkeit] soll genau die Inhalte und die Meinung der Originalarbeit wiedergeben
\end{description}

Diese etwa einseitige Zusammenfassung soll es dem Leser ermöglichen, Inhalt der Arbeit und Vorgehensweise
des Autors rasch zu überblicken. Gegenstand des Abstract sind insbesondere 
\begin{itemize}
\item Problemstellung der Arbeit,
\item im Rahmen der Arbeit geprüfte Hypothesen bzw. beantwortete Fragen,
\item der Analyse zugrunde liegende Methode,
\item wesentliche, im Rahmen der Arbeit gewonnene Erkenntnisse,
\item Einschränkungen des Gültigkeitsbereichs (der Erkenntnisse) sowie nicht beantwortete Fragen. 
\end{itemize}

\end{abstract}
\end{otherlanguage} % auskommentieren, wenn Abstract auf Deutsch sein soll

%%%%%%%%%%%%%%%%%%%%%%%%%%%%%%%%%%%%%%%%%%%%%%%%%%%%%%%%%%%%%%%%%%%%%%%%%%%%%%%%%%%%%%%%%%%%
\newpage
%%%%%%%%%%%%%%%%%%%%%%%%%%%%%%%%%%%%%%%%%%%%%%%%%%%%%%%%%%%%%%%%%%%%%%%%%%%%%%%%%%%%%%%%%%%%

\renewcommand{\abstractname}{\langabstract} % Text für Überschrift
\begin{abstract}

Deutsche Variante des Abstracts.

\end{abstract}
	\newpage

	% Inhaltsverzeichnis
	\begin{spacing}{1.2}
		\begingroup
			% auskommentieren für Seitenzahlen unter Inhaltsverzeichnis
			\renewcommand*{\chapterpagestyle}{empty}
			\pagestyle{empty}

			%\setcounter{tocdepth}{1}
			%für die Anzeige von Unterkapiteln im Inhaltsverzeichnis
			\setcounter{tocdepth}{2}

			%\tableofcontents
			\maintoc
			\clearpage
		\endgroup
	\end{spacing}
	\newpage
    
    \pagestyle{scrheadings}
	
	% noch ausstehende ToDo's - Liste
	% \listoftodos

	% Abkürzungsverzeichnis
 	\clearpage
 	\input{ads/acronyms}

	% Abbildungsverzeichnis
 	\clearpage
	%\addcontentsline{lof}{figure}{\listfigurename}
	\listoffigures
	
	% Tabellenverzeichnis
 	\clearpage
 	\listoftables
	
	% Formelgrößenverzeichnis
 	\clearpage
 	%!TEX root = ../main.tex

\addchap{Formelgrößen}

\begin{spacing}{1.5}
    \begin{tabbing}
        mmmmmmmm \= mmmmmmmm \= \kill
        \textbf{Symbol} \> \textbf{Einheit} \> \textbf{Beschreibung} \\ \vspace*{.5cm} \\
        $K$ \> $\mathsf{m^2}$ \> Permeabilität (Durchlässigkeit) \\
        $p_\mathsf{i}$ \> - \> Punktzahl des Kriteriums i \\
        $p_\mathsf{max}$ \> - \> maximal erreichbare Punktzahl eines Kriteriums i \\
        $p$ \> $\mathsf{\frac{W}{m^2}}$ \> Leistungsdichte \\
        $P$ \> W \> Leistung \\
        $Q$ \> $\mathsf{\frac{m^3}{s}}$ \> Durchflussrate 
    \end{tabbing}
\end{spacing}

	% Quellcodeverzeichnis
	\clearpage
	\lstlistoflistings
	
	\cleardoublepage
	\pagenumbering{arabic}
    
    \pagestyle{scrheadings}		% Kopf- und Fußzeile wie zuvor eingestellt
	
	%\setcounter{footnote}{1}	% Dont start at 2

    \pagestyle{scrheadings}
	% Inhalt
	\foreach \i in {01,02,03,04,05,06,07,08,09,...,99} {%
		\edef\FileName{content/\i kapitel}%
			\IfFileExists{\FileName}{%
				\input{\FileName}
			}
			{%
				%file does not exist
			}
	}

	\clearpage
	
    \pagenumbering{Roman}
    \setcounter{page}{21}
     
	% Literaturverzeichnis
	% \clearpage
	\printbibliography
	% \printbibliography[notkeyword=intern, nottype=patent]
	% \printbibliography[heading=subbibliography, keyword=intern, title={Interne Quellen}]
	% \printbibliography[heading=subbibliography, type=patent, title={Patentschriften}]
	% Unterverzeichnisse siehe: https://texwelt.de/fragen/7532/wie-unterteile-ich-meine-biblatex-bibliografie
    
	% sonstiger Anhang
	\cleardoublepage
	\pagenumbering{Alph}
	\appendix
	% !TeX root = ../dokumentation.tex

\appendixtoc
\renewcommand\thechapter{\Alph{chapter}}
\setcounter{chapter}{0}

% \pagebreak
% \includepdf[pages=-,scale=.9,pagecommand={}]{Aufgabenstellung.pdf} 
% PDF um 10% verkleinert einbinden --> Kopf- und Fußzeile  werden so korrekt dargestellt. Die Option `pages' ermöglicht es, eine bestimmte Sequenz von Seiten (z.B. 2-10 oder `-' für alle Seiten) auszuwählen.
% \pagebreak
%\includepdf[pages=-,scale=.8,pagecommand=\section*{A. eventGenerator.py}]{../appendix/eventGenerator.py.pdf}
%\includepdf[pages=-,scale=.8,pagecommand=\section*{B. sendEvents.py}]{../appendix/sendEvents.py.pdf}


%%%%%%%%%%%%%%%%%%%%%%%%%%%%%%%%%%%%%%%%%%%%%%%%%%%%%
\chapter{Grafiken}

%%%%%%%%%%%%%%%%%%%%%%%%%%%%%%%%%%%%%%%%%%%%%%%%%%%%%
\chapter{Tabellenwerke}

%%%%%%%%%%%%%%%%%%%%%%%%%%%%%%%%%%%%%%%%%%%%%%%%%%%%%
\chapter{Sonstiges}

% \section{Datenblatt: Titan Grade 5 (3.7165; Ti6Al4V) - HWN Titan GmbH \cite{HWN.2022}}
% \label{section:Titandatenblatt}
% \begin{figure}[h]
%     \centering
%     \includegraphics[width=.73\textwidth]{images/appendix/DatenblattTitanGrade5.png}
% \end{figure}
\includepdf[pages=1,scale=.7,pagecommand=\section{Datenblatt: Titan Grade 5 (3.7165; Ti6Al4V) - HWN Titan GmbH}\label{section:Titandatenblatt}]{images/appendix/titan-grade-5-werkstoffdatenblatt.pdf}

\end{document}
