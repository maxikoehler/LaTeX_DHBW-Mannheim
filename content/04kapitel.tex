%!TEX root = ../main.tex

\chapter{Tabellen}

\begin{table}[H]
    \centering
    \small
    \caption{Beispielhafte Tabelle mit einigen Extras}
    \label{tab:tabelle}
    \vspace{12pt}
    \begin{tabularx}{.85\textwidth}{|c|Xr|}
        \cline{2-3}
        \multicolumn{1}{l|}{} & Spalte A & Spalte B \\ \cline{2-3}\noalign{\vskip\doublerulesep\vskip-\arrayrulewidth} \hline 
        \parbox[t]{2mm}{\multirow{3}{*}{\rotatebox[origin=c]{90}{intern}}} 
        & Inhalt 1 & \\
        & & \\
        & & \\ \hline \hline 
        \parbox[t]{2mm}{\multirow{2}{*}{\rotatebox[origin=c]{90}{alternativ}}} 
        & & \\
        & & Inhalt 2 \\ \hline
    \end{tabularx}
\end{table} 

Es sind auch Tabellen über mehrere Seiten und im Querformat möglich. Dabei automatische Spaltenbreite und automatischer Umbruch:
\begin{landscape}
    \LTXtable{24cm}{images/appendix/\jobname-vergleich-ess.tex}
\end{landscape}