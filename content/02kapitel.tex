%!TEX root = ../main.tex

\chapter{Formatierung}

\section{Überschrift Ebene 1}

\subsection{Überschrift Ebene 2}

\subsubsection{Überschrift Ebene 3}
Diese wird nicht mehr im Inhaltsverzeichnis angezeigt.

\section{Formatierung der Schrift}
Der Befehl \verb|\textbf| schreibt ein Text \textbf{fett}.\\
Der Befehl \verb|\textsc| schreibt ein Text \textit{kursiv}.\\

\section{Referenzierungen}\label{sec:referenzierungen}
Grundsätzlich kann in \LaTeX~alles referenziert und gelabled werden.
Dabei müssen \textit{label} immer klein geschrieben werden und darf nur \glqq{}-\grqq{} und \glqq{}:\grqq{} enthalten.\\
Die Referenzierung erfolgt folgendermaßen:\\
Der Befehl \verb|\autoref| referenziert auf den entsprechenden \autoref{sec:referenzierungen}\\
Dabei stellt der Befehl je das Schlüsselwort zur Verfügung, hier: \textit{Abschnitt}.\\ Alternativen sind \textit{Abbildung}, \textit{Tabelle}, \dots

\section{Aufzählungen}
Aufzählungen werden mit folgender Umgebung dargestellt:
\begin{itemize}
    \item Item 1
    \item Item 2
    \begin{itemize}
        \item Unter-Item 1
        \item Unter-Item 2
    \end{itemize}
    \item Item 3
\end{itemize}
Das Ganze geht auch mit Zahlen als Aufzählungen
\begin{enumerate}
    \item Item 1
    \item Item 2
    \begin{enumerate}
        \item Unter-Item 1
        \item Unter-Item 2
    \end{enumerate}
    \item Item 3
\end{enumerate}

\section{Abkürzungen}
Abkürzungen können auch verwendet werden. \acs{KI} oder \acfp{KI}

\section{Formeln}

Formeln können mit \LaTeX~sehr einfach erstellt werden. Diese können mehrzeilig und wahlweise mit Nummerierung als auch Tag (siehe \autoref{eq:formel1}) versehen werden. Das \glqq{}\&\grqq{} bewirkt eine vertikale Ausrichtung der Formeln untereinander an gewählter Stelle. Chemische Reaktionsgleichungen und Sonderzeichen sind ebenso möglich, wie gängige Mathematik Symbole.
\begin{align}
    F&=m \cdot a \notag \\
    \vec{v}_\mathsf{rel}(t)&=\int \vec{a}_\mathsf{rel}(t) \,dx \label{eq:formel1} \\[18pt]
    \mathsf{mit}~[\vec{v}]&=\mathsf{\frac{m}{s}} \notag
\end{align}
Indizes und Einheiten sind grundsätzlich nicht Kursiv und in Schriftart des Textes zu schreiben, Prozentzeichen ebenfalls.