%!TEX root = ../main.tex

\chapter{Sonstiges}

\section{CSV-Plots}

\begin{figure}[H]
    \centering
    \begin{tikzpicture}
        \begin{axis}[
            width=\textwidth,
            height=.6\textwidth,
            grid=both,
            ymin=0 , ymax=850,
            minor y tick num=1,
            xtick={2006,2007,2008,2009,2010,2011,2012,2013,2014,2015,2016,2017,2018,2019,2020,2021,2022},
            tick label style={/pgf/number format/1000 sep=},
            x tick label style={rotate=90},
            xlabel={Jahre}, ylabel={Anzahl},
            legend entries={Legende 1,Legende 2},
            % legend style={at={(0.99,0.99)}, anchor=north west},
            legend pos=north west,
        ]
            \addplot+ [smooth, black!50, mark=square*, mark options={fill=black!50, draw=black!50}] table [x=Jahr,y=Anmeldungen,col sep=semicolon]{images/Zeitverlauf2.csv};
            \addplot+ [smooth, black, mark=square*, mark options={fill=black, draw=black}] table [x=Jahr,y=Erteilungen,col sep=semicolon]{images/Zeitverlauf2.csv};
        \end{axis}
    \end{tikzpicture}
    \caption{Weiterer Beispielplot über csv-Dateien}
    \label{fig:zeitlicher-verlauf}
\end{figure}


\begin{landscape}
    \centering
    \begin{figure}
        \begin{tikzpicture}
            \pgfplotsset{set layers}
            \pgfplotsset{
                width=22cm,
                height=.7\textwidth,
                % legend style={at={(0.99,0.99)}, anchor=north west},
                legend pos=north west,
                legend cell align=left,
                ticklabel shift={0.05cm},
                tick label style={/pgf/number format/1000 sep=},
                xmin=0, xmax=360,
                no markers,
                every axis plot/.append style={thick}
                % enlargelimits=0.05
            }
            \begin{axis}[
                grid=major,
                scale only axis,
                axis y line*=left, % the '*' avoids arrow heads
                ymin=0, ymax=1.8,
                xlabel=x-Achse,
                ylabel=y-Achse 2,
                legend style={at={(0.99,0.99)}, anchor=north east},
                legend entries=Legende 1
            ]
            \addplot+[smooth, dark-green] table[x=A,y=B,col sep=semicolon]{images/Tabelle2.csv};
            \end{axis}
            \begin{axis}[
                scale only axis,
                ymin=0, ymax=250,
                axis y line*=right,
                axis x line=none,
                ylabel=y-Achse 2,
                legend style={at={(0.99,0.92)}, anchor=north east},
                legend entries=Legende 2
            ]
            \addplot+[smooth, darkgray] table[x=A,y=C,col sep=semicolon]{images/Tabelle2.csv};
            \end{axis}
        \end{tikzpicture}
        \caption{Und noch ein weiteres Beipiel eines pgfplots}
    \end{figure}
\end{landscape}

\begin{figure}[h]
    \centering
    \begin{tikzpicture}
        \pgfplotsset{set layers}
        \pgfplotsset{
            width=.8\textwidth,
            height = .3\textheight,
            % legend style={at={(0.99,0.99)}, anchor=north west},
            legend pos=north west,
            legend cell align=left,
            ticklabel shift={0.05cm},
            tick label style={/pgf/number format/1000 sep=},
            xtick={2021,2022,2023,2024,2025,2026,2027,2028,2029,2030},
            no markers,
            enlargelimits=0.05
        }
        \begin{axis}[
            grid=major,
            scale only axis,
            axis y line*=left, % the '*' avoids arrow heads
            ymin=0, ymax=900,
            xlabel=x-Achse,
            ylabel=y-Achse 1,
            legend style={at={(0.01,0.99)}, anchor=north west},
            legend entries=Legende 1
        ]
        \addplot[dark-green] coordinates {
            (2021, 160)(2022, 180)(2023, 220)(2024, 280)(2025, 330)(2026, 400)(2027, 470)(2028, 600)(2029, 720)(2030, 825)
        };
        \end{axis}
        \begin{axis}[
            ybar,
            scale only axis,
            ymin=0, ymax=30,
            axis y line*=right,
            axis x line=none,
            ylabel=y-Achse 2,
            legend style={at={(0.01,0.89)}, anchor=north west},
            legend entries=Legende 2
        ]
        \addplot[black!60, fill=black!30] coordinates {
            (2021, 0.500)(2022, 1.406)(2023, 2.750)(2024, 4.375)(2025, 7.734)(2026, 12.500)(2027, 14.688)(2028, 18.750)(2029, 22.500)(2030, 25.781)
        };
        \end{axis}
    \end{tikzpicture}
    \caption{Geplottetes Balken und Liniendiagramm mit Primär- und Sekundärachse}
    \label{fig:balken-linien-plot}
\end{figure}

\section{Kreisdiagramme}

\begin{figure}[H]
    \centering
    \begin{tikzpicture}
        \pie[rotate=90, style={thin}, radius=3.3, text=legend, color={black!10, black!20, dark-green!70, middle-green!80, light-green!80, black!50, black!60, black!70, black!80, black!90}, before number=\printonlylargeenough{3}, after number=\ifprintnumber\%\fi]
        {5/a , 5/b , 26/c , 50/d , 7/e, 4/f , 3/g}
    \end{tikzpicture}
    \caption{Einfaches Kreisdiagramm}
    \label{fig:kreisdiagramm}
\end{figure}

\begin{figure}[H]
    \centering
    \begin{subfigure}[t]{.49\textwidth}
        \centering
        \begin{tikzpicture}
            \pie[rotate=90, style={thin}, radius=2, color={black!10, black!20, dark-green!70, middle-green!80, light-green!80, black!50, black!60, black!70, black!80, black!90}, before number=\printonlylargeenough{4}, after number=\ifprintnumber\%\fi]
            {12/ , 10/ , 57/ , 0/ , 6/, 9/ , 6/}
        \end{tikzpicture}
        \caption{Struktur A}
        \label{fig:kreis-a}
    \end{subfigure}
    \begin{subfigure}[t]{.49\textwidth}
        \centering
        \begin{tikzpicture}
            \pie[rotate=90, style={thin}, radius=2, text=legend, color={black!10, black!20, dark-green!70, middle-green!80, light-green!80, black!50, black!60, black!70, black!80, black!90}, before number=\printonlylargeenough{4}, after number=\ifprintnumber\%\fi]
            {3/a , 3/b , 58/c , 30/d , 2/e, 2/f , 2/g}
        \end{tikzpicture}
        \caption{Struktur B}
        \label{fig:kreis-b}
    \end{subfigure}
    \caption{Doppeltes Kreisdiagramm mir einer Legende}
    \label{fig:doppeltes-kreisdiagramm}
\end{figure}

\section{TikZ}

\section{Quellcode}

Quellcode kann folgendermaßen dargestellt werden:

\lstinputlisting[caption={Bisheriger Quellcode für die Motor-Regelung auf dem Arduino},captionpos=b,style=style-c,label=lst:arduino-old]{images/codes/arduino-old.c}

Dabei sind die styles \textit{style-c}, \textit{style-cpp} und \textit{style-python} definiert.